\section{Premiações}
    Serão oferecidas premiações às equipes participantes com base em seus méritos individuais demonstrados durante o Kurumin. As premiações são dividas em duas categorias: 
    \begin{enumerate}
        \item Individuais, em que é considerado apenas o esforço da equipe individualmente, havendo critérios claros para a obtenção da premiação;
        \item Coletivos, em que as equipes competirão entre si para determinar a que melhor atende a um dado requisito, a qual será premiada.
    \end{enumerate}   

    Haverão certificados online de participação para cada estudante e Gerente de Projeto participante. Nele, estarão explícitos as conquistas e prêmios que a equipe recebeu, caso aplicável.

    A seguir tem-se o detalhamento das premiações:

    \subsection{Premiações Individuais}
        Forma de premiação dividida em três classes: Geosfera, Troposfera e Estratosfera. A equipe será premiada com uma das conquistas se cumprir todos os seus requisitos.
        
        \textbf{Classe Geosfera}: Para ser elegível a esta condecoração, a equipe deve ter desenvolvido com sucesso a missão proposta dentro dos requisitos impostos neste edital, redigindo todos os relatórios solicitados conforme as exigências estabelecidos para cada um e dentro dos respectivos prazos.
        
        \textbf{Classe Troposfera}: Além de cumprir com as exigências da Classe Geosfera, o projeto desenvolvido pela equipe deve apresentar resultados de grande relevância científica ou social, explicitando-os adequadamente nos relatórios entregues. Ademais, a equipe deve apresentar seu projeto no Sábado Aeroespacial ou enviar um vídeo da apresentação, como destacado na Seção Lançamento.
        
        \textbf{Classe Estratosfera}: Cumpridos os requisitos das classes Geosfera e Troposfera, o experimento deve possuir uma metodologia técnica e científica robusta. A equipe deve ter planejado um experimento que tire completo proveito das situações encontradas na estratosfera, desenvolvido um conjunto experimental de excelência para a missão proposta, e analisado as informações coletadas durante o lançamento de forma metódica e cuidadosa. Equipes nesta classe apresentam um experimento digno de publicação.
        
    \subsection{Premiações Gerais}
        Nesta classificação, apenas uma equipe de cada etapa de ensino será nomeada para cada premiação:
        \begin{itemize}
            \item \textbf{Melhor Proposta Experimental}: Será premiada a equipe a desenvolver a melhor Proposta Experimental, planejando sua missão de modo excepcional;
            \item \textbf{Melhor Relatório Parcial}: Será premiada a equipe a desenvolver o melhor Relatório Parcial, desenvolvendo sua missão de modo excepcional;
            \item \textbf{Melhor Relatório Final}: Será premiada a equipe a desenvolver o melhor Relatório Final, executando sua missão e analisando seus resultados de modo excepcional;
            \item \textbf{Melhor Metodologia Experimental}: Será premiada a equipe a apresentar a melhor Metodologia Experimental, definindo cuidadosamente os materiais e os métodos para o cumprimento da missão, com excelência técnica e científica;
            \item \textbf{Maior Relevância Científica}: Será premiada a equipe a desenvolver a missão com maior relevância científica, comprovada por meio de seus relatórios;
            \item \textbf{Maior Relevância Social}: Será premiada a equipe a desenvolver a missão com maior relevância social, comprovada por meio de seus relatórios;
            \item \textbf{Experimento mais Inovador}: Será premiada a equipe a propor e desenvolver a missão mais inovadora, surpreendo os avaliadores em sua criatividade e ingenuidade;
            \item \textbf{Melhor Montagem Experimental de PseudoSat}: Será premiada a equipe de PseudoSat a desenvolver a melhor montagem experimental, demonstrando maestria prática no planejamento e montagem do conjunto;  
            \item \textbf{Melhor Engenharia de FullSat}: Será premiada a equipe de FullSat a apresentar a melhor arquitetura de sistema, desenvolvida com excelência com base nos requisitos do Sat e da missão;
            \item \textbf{Grande Prêmio}: Por fim, o prêmio de maior prestígio do Kurumin será dados à equipe que tenha demonstrando excelência técnica e científica no planejamento, execução e análise dos resultados de seus projetos, desenvolvendo com competência excepcional sua missão e registrando-a nos relatórios com clareza e distinção.
        \end{itemize}
        
        A equipe vencedora do Grande Prêmio será premiada com o Troféu do Grande Prêmio da Missão Kurumin 2024.