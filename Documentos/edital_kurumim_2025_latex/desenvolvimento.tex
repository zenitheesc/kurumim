\section{Fase 2 - Desenvolvimento}
    Após a submissão do relatório, o grupo Zenith divulgará no site e nas suas mídias oficiais as equipes aprovadas para a Fase 2. A aprovação também será comunicada diretamente aos respectivos Gerentes de Projeto, por e-mail, acompanhado do \textit{feedback} da equipe de correção com sugestões para a melhoria do projeto. Conforme necessário, será enviado conjuntamente um boleto para o pagamento das devidas taxas, conforme discutido na Seção 1.3.

    As equipes são incentivadas a iniciar o desenvolvimento do projeto proposto antes mesmo de sua aprovação pela comissão avaliadora, proporcionando um tempo maior para a criação de um projeto de excelência.
    
    Ademais, durante o desenvolvimento do projeto, 
    diversos novos desafios surgirão e serão enfrentados pelo grupo para 
    garantir seu sucesso. Portanto, nesta fase, um novo relatório deverá 
    ser redigido e enviado ao Zenith, trazendo em detalhes o projeto final 
    escolhido (incluindo quaisquer mudanças), como este foi desenvolvido e 
    testes realizados para assegurar seu correto funcionamento durante o 
    lançamento. Este relatório, o respectivo experimento e sua ficha de dados 
    devem ser entregues ao Zenith até a data limite de 01/10.
    
    A seguir são detalhados os relatórios específicos a cada modalidade (PseudoSat e FullSat).

    \subsection{Relatório Parcial - PseudoSat}
        Equipes participantes na categoria PseudoSat devem seguir este modelo de relatório para descrever o experimento desenvolvido. Este deverá possuir, no máximo, 10 páginas.
        
        \subsubsection{Capa}
            O Relatório deve contar com uma capa semelhante à Proposta enviada:
            \begin{enumerate}
                \item Título do Trabalho; 
                \item Nome das Instituições de Ensino; 
                \item Nome da Equipe; 
                \item Nome dos Integrantes; 
                \item Nome do Gerente de Projeto;
                \item Nível de Ensino da Equipe;
                \item Local, Mês e Ano. 
            \end{enumerate}
                    
        \subsubsection{Resumo}
            Novamente deve ser oferecido um resumo atualizado do projeto desenvolvido, cobrindo quaisquer mudanças em relação à proposta original. Até 200 palavras.

        \subsubsection{Missão}
            Deve-se aqui declarar a missão do experimento desenvolvido, delineando qual é seu objetivo e como este será cumprido. É imprescindível destacar os méritos científicos ou sociais da missão.
        
        \subsubsection{Metodologia}
            Esta seção objetiva explicar de forma detalhada e exata os métodos e técnicas empregados no desenvolvimento do experimento realizado no satélite, cobrindo a escolha de materiais ou equipamentos, a montagem experimental, a análise e processamento de dados, e qualquer outro item pertinente ao experimento.

        \subsubsection{Lista de materiais}
            Complementando a seção de metodologia, deve ser anexada uma lista dos materiais empregados no desenvolvimento do experimento como um todo. Deve-se destacar ao final a massa total do conjunto experimental, respeitando a seção de Requisitos.

        \subsubsection{Integridade Estrutural}
            Deve-se detalhar a estrutura do experimento de fixação do experimento, garantindo sua rigidez e segurança durante o voo e, especialmente, durante a queda da sonda. Experimentos que não possuam uma estrutura adequada podem ser perdidos ou destruídos durante o lançamento.

        \subsubsection{Referências}
            Quaisquer referências consultadas durante o desenvolvimento da missão devem ser aqui mencionadas.

    \subsection{Relatório Parcial - FullSat}
        Equipes participantes na categoria FullSat devem seguir este modelo de relatório para descrever o satélite desenvolvido. Este deverá possuir, no máximo, 20 páginas.

        Ressaltamos que os subsistemas podem ser desenvolvidos completamente pelas equipes ou ainda serem compostos parcialmente por modelos comerciais (como os satélites educacionais da Pyon).
        
        \subsubsection{Capa}
            O Relatório deve contar com uma capa semelhante à Proposta enviada:
            \begin{enumerate}
                \item Título do Trabalho; 
                \item Nome das Instituições de Ensino; 
                \item Nome da Equipe; 
                \item Nome dos Integrantes; 
                \item Nome do Gerente de Projeto;
                \item Nível de Ensino da Equipe;
                \item Local, Mês e Ano. 
            \end{enumerate}
            
        \subsubsection{Resumo}
            Novamente deve ser oferecido um resumo atualizado do projeto desenvolvido, cobrindo quaisquer mudanças em relação à proposta original. Até 200 palavras.
        
        \subsubsection{Missão}
            Deve-se aqui declarar a missão do satélite desenvolvido, delineando qual é seu objetivo e como este será cumprido. É imprescindível destacar os méritos científicos ou sociais da missão.
        
        \subsubsection{Metodologia}
            Esta seção objetiva explicar de forma detalhada e exata os métodos e técnicas empregados no desenvolvimento do experimento realizado no satélite, cobrindo a escolha de materiais ou equipamentos, a montagem experimental, a análise e processamento de dados, e qualquer outro item pertinente ao experimento.

        \subsubsection{Lista de materiais}
            Complementando a seção de metodologia, deve ser anexada uma lista dos materiais empregados no desenvolvimento do experimento e do satélite como um todo. Deve-se destacar ao final a massa total do conjunto experimental, respeitando a seção de Requisitos.
        
        \subsubsection{Arquitetura do Sistema}
            Nesta seção deve-se incluir uma descrição completa do design do sistema do satélite, descrevendo o desenvolvimento e funcionamento de todos os subsistemas essenciais ao pleno funcionamento do satélite e subsistemas necessários para a execução da missão proposta.

            Deverão ser anexados ao longo do relatório os projetos mecânicos e eletrônicos correspondentes aos subsistemas, incluindo também um fluxograma de como este foi programado para executar sua respectiva função.
            
            Ao final, a arquitetura geral para a integração e controle dos subsistemas deve ser detalhada, demonstrando como este sistema completo concluirá a missão proposta.

            Códigos e desenhos técnicos devem ser incluídos ao final do relatório na seção de apêndices.
        
        \subsubsection{Testes e Simulações}
            De forma a assegurar o correto funcionamento durante o voo, é essencial aplicar sobre ele um conjunto de testes ainda em solo, além de simular seu comportamento na alta atmosfera.

            Portanto, requisita-se nesse seção a inclusão do resultado de uma bateria de testes, detalhando como estes foram executados e sob quais parâmetros:
            \begin{enumerate}
                \item Sensoriamento interno;
                \item Integridade estrutural;
                \item Integridade térmica;
                \item Integridade eletromagnética;
                \item Comunicação.
            \end{enumerate}

            Testes adicionais podem ser incluídos a critério da equipe, caso incrementem a confiabilidade do satélite.

        \subsubsection{Referências}
            Quaisquer referências consultadas durante o desenvolvimento da missão devem ser aqui mencionadas.

    \subsection{Envio dos Projetos}
        Ao finalizar o desenvolvimento do projeto, o grupo deverá enviá-lo ao Zenith para 
        lançamento, devendo chegar ao grupo até a data limite de 01/10/2025. 
        Solicitamos que o código de rastreio do experimento seja enviado por 
        e-mail ao grupo, para acompanhar seu progresso. O Zenith não se responsabilizará 
        por atrasos ou danos durante a entrega.
        
        O endereço de entrega será comunicado por e-mail exclusivamente ao Gerente de Projeto. Destaca-se que cada equipe ficará responsável pelo custo do transporte do experimento à sede do Zenith em São Carlos - SP.
    
    \subsection{Ficha de Dados}
        Também para ser entregue junto com os experimentos, temos a Ficha de Dados, documento criado com o intuito de objetivar e facilitar a passagem de informações entre os alunos e os membros do Zenith quanto aos experimentos a serem lançados. Ele deve ser preenchido e enviado posteriormente ao e-mail do Zenith até a data limite de envio dos experimentos. Este documento foi feito em formato PDF e está disponível no site do Zenith, sendo imprescindível que esteja completamente preenchido para que o experimento seja lançado.
        
        Nele, a equipe deve fornecer informações gerais do experimento, de maneira clara e objetiva. Além dos itens nome da equipe, nível de ensino e tipo de sat, existem:

        \subsubsection{Massa Total}
            Deve ser informada à massa total do conjunto, contando experimento, invólucro, tampas e qualquer outro material a ser lançado. Para tanto, a equipe deve pesar todo o material empregado no experimento e em sua contenção conjuntamente em uma balança de precisão suficiente para a incerteza permitida: 5 g para o SmallSat e de 10 g para o Big-Sat. Observe a massa máxima permitida na Seção de Requisitos.
            
            Este quesito é extremamente importante para a manutenção do projeto, por conta das legislações a serem cumpridas para o lançamento da sonda final. Por isso, experimentos que não cumprirem com este requisito poderão ser eliminados do projeto mesmo após seu desenvolvimento e chegada no laboratório do Zenith.

        \subsubsection{Manejo e Manutenção do experimento}
            
            Neste tópico, os participantes devem descrever detalhadamente, explicando os processos e datas referentes, como seu experimento deve ser mantido e cuidado desde a data de recebimento do SAT até aos dias anteriores ao lançamento, caso sejam necessários cuidados especiais.
            
            A equipe poderá solicitar nesta seção uma vídeo chamada com o Zenith na semana anterior ao lançamento, com data e hora a serem estipulados. As chamadas possibilitarão que a equipe instrua um membro do grupo Zenith a fazer alguma pequena manutenção no experimento antes do lançamento, como a troca de uma placa de petri ou a troca de uma bateria, sem alteração considerável na massa ou volume do SAT.

        \subsubsection{Recomendações na fixação à sonda}
            
            Casos haja algum cuidado especial na fixação do experimento à sonda, este deve ser aqui detalhado. Deve-se indicar, por exemplo, a extensão máxima de parafuso para fixação, no caso de FullSats, ou especificar a necessidade de posicionar o experimento na sonda com uma dada orientação.

        \subsubsection{Manuseio do Experimento}
            
            Caso haja a necessidade de qualquer manejo minutos antes do lançamento ou após o resgate da sonda, a equipe deve esclarecer como se dará esse processo. É importante detalhar o procedimento e material que deve ser utilizado, lembrando que caso seja necessária uma instrumentação específica, esta terá de ser enviada juntamente com o experimento para o Zenith.
        
            Caso a equipe necessite visitar o laboratório do Zenith para algum ajuste no experimento, deve deixar claro os motivos pelos quais isso será necessário e o que será feito.
            
            Algumas observações pertinentes nessa seção são:
            \begin{itemize}
                \item Observar se existe algum sistema (botão e/ou interruptor) que deve ser acionado; 
                \item Detalhar condições de armazenamento e transporte (sensibilidade à temperatura, umidade, pressão, fragilidade a impactos, etc.) e proximidade de objetos/instrumentos sensíveis a campos eletromagnéticos (passíveis de distorções, estrago);
                \item Detalhar (se houver) diferença de manuseio do experimento antes e depois do voo.
            \end{itemize}

    \subsection{Avaliação - PseudoSat}
        A seguir trazemos os critérios de avaliação para o Relatório Parcial da modalidade \textbf{PseudoSat} e seus respectivos pesos. Certos critérios de avaliação permanecem os mesmos, permitindo às equipes iterar sobre o projeto original com o feedback fornecido na fase anterior.
        
        \begin{itemize}
            \item \textbf{Clareza}: Serão avaliadas a organização, clareza e objetividade do relatório (peso 2);
                     
            \item \textbf{Missão}: Serão novamente avaliados os objetivos da missão e sua motivação, incluindo sua relevância científica e social (peso 2);
            
            \item \textbf{Metodologia}: Será avaliada a metodologia de desenvolvimento prático do projeto, a qual deve possuir fundamentação coerente com os objetivos, estar clara e compreensível (peso 4);

            \item \textbf{Integridade Estrutural:} Será avaliado se foram feitas considerações adequadas acerca da integridade estrutural do experimento e sua correta fixação, de forma a prevenir falhas (peso 1).
        \end{itemize}
        
        A cada um desses critérios será atribuída uma nota de 1 a 10, ponderando de acordo com os pesos fornecidos para a média final.

    \subsection{Avaliação - FullSat}
        A seguir trazemos os critérios de avaliação para o Relatório Parcial da modalidade \textbf{FullSat} e seus respectivos pesos. Certos critérios de avaliação permanecem os mesmos, permitindo às equipes iterar sobre o projeto original com o feedback fornecido na fase anterior.
        
        \begin{itemize}
            \item \textbf{Clareza}: Serão avaliadas a organização, clareza e objetividade do relatório (peso 2);
                     
            \item \textbf{Missão}: Serão novamente avaliados os objetivos da missão e sua motivação, incluindo sua relevância científica e social (peso 2);
            
            \item \textbf{Metodologia}: Será avaliada a metodologia de desenvolvimento prático do projeto, a qual deve possuir fundamentação coerente com os objetivos, estar clara e compreensível (peso 4);

            \item \textbf{Arquitetura de Sistema:} Será avaliada a Arquitetura desenvolvida e seu funcionamento, determinando se todos os subsistemas essenciais à missão foram devidamente desenvolvidos e integrados (peso 5);

            \item \textbf{Testes e Simulações:} Serão avaliadas a representatividade dos testes e simulações em relação à realidade, o sucesso do satélite nos testes propostos, e a cobertura dos testes sobre seus subsistemas (peso 3).
        \end{itemize}
        
        A cada um desses critérios será atribuída uma nota de 1 a 10, ponderando de acordo com os pesos fornecidos para a média final.