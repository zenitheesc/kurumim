\section{Fase 1 - Planejamento}\label{sec:planejamento}
    \par A Fase de Planejamento do Kurumim será constituída por dois documentos a serem 
    enviados pelas equipes: sua Ficha de Inscrição (item \ref{sec:ficha_inscricao}), preenchida por meio do formulário 
    presente no site até 21/07, e sua Proposta Experimental, enviada por e-mail (zenith@usp.br) 
    até a data limite 04/08, para avaliação.

    A seguir ambos os documentos serão detalhados, fornecendo-se também os critérios de avaliação para a Proposta Experimental.

    \subsection{Ficha de Inscrição}\label{sec:ficha_inscricao}
        O intuito deste documento é conter dados gerais acerca da equipe e seus integrantes, além da modalidade selecionada para competição.
    
        \subsubsection{Nome da Equipe}
            A equipe deve escolher um nome que a represente. Durante a divulgação das equipes aprovadas e demais publicações será divulgado somente o nome da equipe, não incluindo seus integrantes.
            
        \subsubsection{Participantes}
            Os alunos devem formar equipes de 2 a 5 estudantes, todos devidamente matriculados em uma instituição de ensino no ano de 2024. Observamos também que não será permitida a participação de um aluno em mais de uma equipe.

            A respectiva categoria (C1, C2 ou C3) da equipe deve ser indicada no ato da inscrição, destacando que esta será definida pelo integrante com maior nível de escolaridade. 
                
                % \item Ao se inscrever no projeto, os alunos autorizam o grupo Zenith a divulgar a sua imagem e a da equipe. Por esse motivo, cita-se que todo o conteúdo desenvolvido pelos estudantes, apesar de ser propriedade intelectual da equipe, poderá ser utilizado para fins de divulgação pelo grupo Zenith respeitando-se a identidade dos participantes.
    
        \subsubsection{Gerente de Projeto}
            Todas as equipes inscritas devem contar com um gerente de projeto, 
            responsável pela inscrição da equipe e por intermediar o contato 
            com o grupo Zenith Aerospace. As seguintes orientações devem ser observadas:
            
            \begin{itemize}
                \item Equipes C1 ou C2 devem contar com um Gerente de Projeto maior de 18 anos, externo à mesma. Preferencialmente deve ser um professor, diretor ou responsável por um dos alunos.
                \item  Equipes C3 devem definir um de seus membros como o Gerente de Projeto.
            \end{itemize}
    
            Os Gerente devem se atentar às regras e prazos descritos neste Edital, 
            além de futuras comunicações oficiais. O contato entre Zenith e Gerente
            será feito unicamente via e-mail (zenith@usp.br).
            
            Qualquer informação que afete o Kurumim de forma geral será 
            divulgada diretamente aos gerentes e também pelas mídias 
            oficiais do Zenith: \href{https://zenith.eesc.usp.br/kurumim}{site} e \href{https://www.instagram.com/zenith\_eesc/}{Instagram}.
    
        \subsubsection{Instituição de Ensino}
            No ato da inscrição, deverão ser fornecidos os dados da instituição de ensino correspondente a cada integrante do grupo. Reitera-se que a instituição de ensino, se brasileira, deve ser oficialmente reconhecida junto ao MEC.
    
            Observe que equipes com participantes de diferentes instituições são permitidas, especificando as instituições correspondentes.

        \subsubsection{Modalidade Experimental}
            Neste momento, os participantes devem especificar a modalidade escolhida para seus experimentos, conforme detalhado na Seção 1.1. Cada equipe deve selecionar uma, e somente uma modalidade: PseudoSat \textbf{ou} FullSat.

    \subsection{Proposta Experimental}\label{sec:proposta_experimental}
        Somando-se à Ficha de Inscrição, a equipe deve redigir uma Proposta Experimental, contendo todas as informações e o embasamento por trás do projeto. Sugere-se que este seja feito de acordo com as normas da ABNT (Associação Brasileira de Normas Técnicas), sendo que as equipes poderão fazer seu relatório sobre o template disponibilizado pelo Zenith no \href{https://github.com/zenitheesc/kurumim/tree/main/Relat%C3%B3rios}{\color{highcolor} github do Kurumim}.

        Os projetos devem ser de autoria própria dos membros da equipe. 
        Qualquer conteúdo de outros autores e/ou fontes, deverá ser devidamente referenciado e identificado. 
        Caso o conteúdo não seja referenciado, o trabalho será considerado plágio e estará sujeito à 
        desclassificação. É esperado ainda que o desenvolvimento seja escrito e desenvolvido exclusivamente 
        pelos alunos participantes, fugindo ao intuito do projeto que este seja desenvolvido por terceiros 
        (incluindo os gerentes de equipes C1 e C2), assim como seja identificado o uso excessivo de ferramentas
        de Inteligência Artificial Generativa para a geração do relatório. Isto é válido para todos os 
        relatórios presentes neste Edital.

        Lembre-se que o experimento proposto deve atender a Seção de Requisitos, ao final deste edital, 
        em sua totalidade, sob pena de desclassificação do projeto. Atente-se especialmente à massa e às 
        dimensões do projeto, caso o experimento entregue para o lançamento do Sábado Aeroespacial seja superior
        ao informado no relatório parcial, a equipe pode sofrer penalizações como o não embarque do exeperimento 
        na sonda, pois essas diferenças de peso podem ser prejudiciais para o lançamento da sonda.

        O projeto deve conter no máximo 10 páginas, sem contar com capa e referências, possuindo os seguintes itens:
        
        \subsubsection{Capa}
            A proposta deve iniciar com uma capa contendo suas informações gerais, como enumerado abaixo:
            \begin{enumerate}
                \item Título do Trabalho; 
                \item Nome das Instituições de Ensino; 
                \item Nome da Equipe; 
                \item Nome dos Integrantes; 
                \item Nome do Gerente de Projeto;
                \item Nível de Ensino da Equipe;
                \item Modalidade escolhida;
                \item Local, Mês e Ano. 
            \end{enumerate}
                    
        \subsubsection{Resumo}
            O resumo é o primeiro tópico a ser tratado na Proposta Experimental, sendo uma apresentação sucinta e objetiva do projeto a ser desenvolvido, seus objetivos e resultados esperados. Até 200 palavras.
                
        \subsubsection{Missão}
            Esta seção visa apresentar o projeto e contextualizá-lo na atualidade, apresentando os objetivos da equipe com o projeto e como ele visa cumprir tal objetivo. Devem ser detalhadas também as motivações que fomentaram a escolha da missão dada, além de sua relevância social e/ou acadêmica.
                
        \subsubsection{Metodologia}
            Neste tópico deve-se discorrer detalhadamente sobre todas as atividades que serão feitas, apresentando com clareza os materiais a serem utilizados e os procedimentos a serem executados, de forma a assegurar o sucesso da missão. Como tópicos obrigatórios para esta seção tem-se:
            \begin{enumerate}
                \item Procedimentos que o grupo pretende realizar para testar suas hipóteses;
                \item Explicação dos procedimentos empregados na confecção do experimento e sua análise;
                \item Elaboração de um orçamento total do projeto.
            \end{enumerate}

            Pedimos que as equipes informem nesta seção qual tamanho de experimento irão selecionar em sua modalidade, respeitando a Seção de Requisitos, ao levar em conta seus objetivos e possíveis restrições técnicas ou monetárias.
            \begin{enumerate}
                \item Equipes da modalidade PseudoSat deverão escolher entre SmallSat ou BigSat;
                \item Equipes da modalidade FullSat deverão escolher entre CanSat ou CubeSat.
            \end{enumerate}

        \subsubsection{Lista de Materiais}
            Complementando a seção de metodologia, deve ser anexada uma lista dos materiais empregados no desenvolvimento do experimento e do satélite como um todo. Deve-se destacar ao final a massa total do conjunto experimental, respeitando a seção de Requisitos.

        \subsubsection{Arquitetura de Sistema}
            \textbf{Apenas para inscritos na categoria FullSat.}
            
            Nesta seção deve-se incluir uma descrição completa do design do sistema do satélite, focando em dois eixos centrais:
            \begin{enumerate}
                \item Descrever todos os subsistemas essenciais ao pleno funcionamento do satélite e subsistemas necessários para a execução da missão proposta;
                \item Definir uma arquitetura para a integração e controle dos subsistemas.
            \end{enumerate}

           Mais detalhes sobre o desenvolvimento de nanosatélites podem ser obtidos no documento \href{https://obsat.org.br/ebook/download.php?file=eBook-PequenosSatelites-Dez-2022.pdf}{\color{highcolor}"Pequenos satélites: Grandes possibilidades"}, produzido em co-autoria com o Zenith, enquanto um exemplo de relatório pode ser encontrado no \href{https://github.com/zenitheesc/USPSat-I/blob/master/USPSat_Report.pdf}{\color{highcolor}repositório do Zenith}.
            
            Destaca-se que esta é uma proposta inicial, que será trabalhada e expandida futuramente.
            
        \subsubsection{Cronograma}
            Detalhamento das datas de atividades de planejamento, desenvolvimento e análise do 
            projeto em forma de tabela, levando em conta o cronograma geral do Kurumim.
                
        \subsubsection{Resultados Esperados}
            O penúltimo item deve conter um resumo de quais resultados se esperam para o projeto após o voo.
                    
        \subsubsection{Referências}
            Bibliografia dos recursos utilizados.

    \subsection{Avaliação}\label{sec:avaliacao_planejamento}
        A seguir discernem-se os critérios de avaliação para a Proposta Experimental e seus respectivos pesos. A pontuação aqui obtida determinará as equipes que poderão avançar na competição.
        
        \begin{itemize}
            \item \textbf{Clareza}: Serão avaliadas a organização, clareza e objetividade da proposta (peso 2);
            
            \item \textbf{Inovação}: Será avaliado o teor de inovação do projeto, através do desenvolvimento de uma nova ideia, métodos usados, entre outros (peso 1);
                     
            \item \textbf{Missão}: Serão avaliados os objetivos da missão e sua motivação, incluindo sua relevância científica e social (peso 2);
            
            \item \textbf{Metodologia}: Serão avaliadas a metodologia científica e formulação teórica do projeto. A metodologia deve possuir fundamentação teórica coerente com os objetivos, estar clara e compreensível (peso 4).
        \end{itemize}

        \textbf{Especificamente à categoria FullSat}, será avaliada também a Arquitetura de Sistema proposta, determinando seu conhecimento técnico acerca de satélites:        
        \begin{itemize}
            \item \textbf{Arquitetura de Sistema:} Será avaliada a Arquitetura desenvolvida e sua aplicabilidade, determinando se todos os subsistemas essenciais à missão foram devidamente identificados e detalhados (peso 4). 
        \end{itemize}
        
        A cada um desses critérios será atribuída uma nota de 1 a 10, ponderando de acordo com os 
        pesos fornecidos para a média final.
