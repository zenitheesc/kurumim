\section{Fase 3 - Lançamento}\label{sec:lancamento}
    \par Concluídas todas as etapas de desenvolvimento e preparação dos experimentos, 
    resta o lançamento. Este se dará primariamente no Campus 2 da USP de São Carlos no 
    dia 04/10, no evento do Sábado Aeroespacial, ao qual convidamos todos os participantes.
    A seguir é detalhado o evento:    

    \subsection{Sábado Aeroespacial}\label{sec:sabado_aeroespacial}
        Anualmente, o grupo Zenith Aerospace oferta à comunidade o evento de divulgação científica nomeado Sábado Aeroespacial. O intuito do evento é proporcionar uma experiência instrutiva e inspiradora aos participantes da missão Kurumim, trazendo uma série de palestras, workshops e minicursos para enriquecer a conhecimento dos participantes. Outros grupos extracurriculares e científicos serão convidados ao evento para oferecer uma experiência ampla e multidisciplinar.

        Detalhes sobre o evento, como palestrantes, minicursos e parcerias serão divulgadas nas redes sociais do grupo Zenith. Todos os participantes da olimpíada são convidados e incentivados a comparecer.
        
        O lançamento das sondas que levarão os SATs desenvolvidos será parte do evento, proporcionando um contato direto com o trabalho desenvolvido pelo Zenith, além da participação direta das equipes nessa etapa. Em caso de força maior, por condições meteorológicas ou não autorização de uso do espaço aéreo, pode ser que a data do lançamento divirja do evento.

    \subsection{O Lançamento}\label{sec:lançamento}
        Espera-se que o lançamento faça parte do evento Sábado Aeroespacial. No entanto, por razões já citadas, o lançamento pode ser remarcado, comunicando-se aos Gerentes de Projeto qualquer eventualidade.

        Destaca-se que o lançamento é, por natureza, um teste rigoroso sobre os projetos desenvolvidos. O Zenith não se responsabiliza por qualquer dano aos experimentos.