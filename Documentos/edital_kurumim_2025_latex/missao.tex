\section{A Missão} 
    \par A Missão Kurumim é a vertente educacional do grupo Zenith, tendo por objetivo promover e difundir a ciência e engenharia aeroespacial através de experimentos técnicos e científicos lançados nas sondas estratosféricas do grupo. O ambiente estratosférico, devido à suas características ambientais, é perfeito para a emulação de condições espaciais, com radiação, temperatura, pressão e altitude extremas, configurando um excelente ambiente de testes.
    
    \par Portanto, através do Kurumim, é oferecido aos estudantes dos ensinos Fundamental, Médio, Técnico e Superior, de instituições de ensino públicas e privadas, a oportunidade de desenvolver experimentos nos mais diversos campos do conhecimento (como Biologia, Física, Eletrônica, Telecomunicações, dentre outros) a serem realizados na estratosfera, buscando inspirá-los a seguir o rumo da ciência e engenharia, além envolver-se no setor Aeroespacial.
    
    \par A missão é totalmente desenvolvida pelo grupo Zenith Aerospace, um grupo de extensão da 
    Escola de Engenharia de São Carlos (EESC-USP), formado por alunos de graduação e pos-graduação da USP
    e da UFSCAR, que tem por único intuito fomentar a produção científica realizada por seus membros e o 
    desenvolvimento da engenharia aeroespacial no país, sem fins lucrativos. 
    A olimpíada é realizada em sua maior parte de forma não presencial, com o envio de relatŕios e \emph{feedbacks}
    acontecendo via formulário e e-mail. Após a seleção dos projetos que estarão na sonda a qual será enviada a estratosfera
    , cujos critérios e formas de envio e formas de envio dos projetos estão descritos neste edital. A missão
    inclui um evento presencial, o \textbf{Sábado Aeroespacial}, onde é realizado o lancamento da sonda, além
    de abranger atividades durante todo o dia com as equipes participantes.
    
    \par Nesta oitava edição do projeto, teremos como tema \textbf{Adapatação ao Aquecimento Global}, 
    e título oficial \textbf{Missão Kurumim: Mudanças Climéticas e o Futuro da Terra}, dando um enfoque
    às mudanças climáticas que estamos vivenciando e a realização da 30ª Conferência das Partes (COP30) em Belém do Pará, no Brasil.
    Incentivamos os participantes a desenvolverem soluções e explorar o tema dando espaço a discussão sobre
    o que podemos fazer para mitigar e como lidar com as mudanças as quais tornam-se cada vez mais drásticas
    e alarmantes.
    
    \par Abaixo listamos alnguns tópicos a serem explorados envolvendo o tema, no entanto, a criatividade
    e novas ideias são fortemente incentivadas:

    \begin{itemize}
        \item Monitoramento da acidificação dos oceanos devido ao CO2;
        \item Adaptação de espécies para sobreviverem às mudanças climáticas;
        \item Monitoramento da atmosfera com balões e satélites;
        \item Pesquisas de adaptar espécies da terra para outros planetas;
        \item A era do Antropoceno como consequência do descuido histórico com o meio-ambiente;
        \item Monitoramento via satélite da adaptação da fauna e flora brasileiras às mudanças climáticas;
        \item Captura de carbono a partir de espécies no espaço;
        \item Tecnologias espaciais para ajudar no combate às mudanças climáticas;
        \item Uso de energia limpa para fazer pesquisas científicas aeroespaciais;
        \item Efeitos da geração de lixo espacial;
        \item Desenvolvimento e melhoramento de sensores e sistema de monitoramento de gases do efeito estufa;
        \item Desenvolvimento de materiais sustentáveis e adaptáveis.
    \end{itemize}

    \par São diversas vertentes de experimentação possíveis: 

    
    \subsection{Modalidades da Olimpíada} \label{sec:modalidades}
        Para facilitar a leitura do edital, a partir deste momento iremos dividir as equipes inscritas em três categorias com base em nível de escolaridade:
        \begin{enumerate}
            \item \textbf{C1:} Ensino Fundamental (I e II);
            \item \textbf{C2:} Ensino Médio ou Técnico;
            \item \textbf{C3:} Ensino Superior.
        \end{enumerate}

        \par Observe que, para os fins deste edital, o nível de escolaridade corresponde ao que o participante está cursando atualmente. Um aluno no 1º do Ensino Médio corresponderá à categoria C2, por exemplo.
    
        \par A olimpíada é constituída por diferentes modalidades, as quais podem ser selecionadas de acordo com o nível de escolaridade da equipe. Sendo assim, definiremos agora as modalidades em que cada um poderá competir.
        
        \begin{itemize}
            \item \textbf{PseudoSat:} Desenvolvimento de pequenos experimentos individuais abertos à estratosfera, em que o foco se dará exclusivamente no experimento proposto e seu estudo, disponível em dois tamanhos a depender das necessidades da equipe: SmallSat e BigSat. Aberto a todos os níveis de ensino (\textbf{C1, C2 e C3});
            
            \item \textbf{FullSat:} Desenvolvimento de um sistema completo para experimentação na forma de um Cansat ou CubeSat, a depender das necessidades da equipe, incluindo todos os subsistemas necessários para o lançamento em órbita (como comunicação, energia, GPS, sensoriamento, etc). Aberto a todos os níveis, mas pedimos que equipes C1 pensem no alto nível de complexidade envolvido antes de escolher a modalidade.
        \end{itemize}

        Ambas objetivam levar aos alunos uma experiência interdisciplinar e incentivar seu desenvolvimento técnico e científico, além de trazer um enfoque para as diferentes possibilidades de estudo do setor aeroespacial.

    \subsection{Fases}
        A Olimpíada será dividida em 4 fases distintas, cobrindo a concepção do experimento, seu desenvolvimento, sua implementação e a análise final dos dados. Abaixo temos as fases detalhadamente:
        \begin{enumerate}
            \item \textbf{Fase 1 - Planejamento:} Devem ser aqui definidas as equipes e seus nomes, entregando uma visão geral do projeto que será desenvolvido e sua motivação;
            \item \textbf{Fase 2 - Desenvolvimento:} O projeto detalhado na inscrição deve agora ser plenamente desenvolvido, entregando um relatório completo da metodologia empregada;
            \item \textbf{Fase 3 - Lançamento:} Com o experimento construído, este deve ser entregue ao grupo para o lançamento em uma de nossas sondas;
            \item \textbf{Fase 4 - Resultados:} A partir dos resultados obtidos durante o lançamento do experimento, deve ser redigido um relatório final trazendo suas conclusões.
        \end{enumerate}

    \subsection{Doações e Patrocínios}
        \par Como já destacado, a Missão Kurumim é inteiramente organizada pelo grupo Zenith 
        com intuito de divulgar e promover tecnologias aeroespaciais, sem qualquer fim lucrativo.

        \par Portanto, para a realização dos lançamentos necessários, altamente custosos, 
        o grupo é dependente de doações e patrocínios para arcar com os custos.
        
        \par Caso a campanha e patrocínios não cubram os custos previstos, 
        serão cobradas as seguintes taxas de grupos classificados para a Fase 2 da competição, 
        sendo necessário seu pagamento para a continuidade da equipe na 
        Missão (informamos que grupos de \textbf{escolas da rede pública estarão isentos} desta taxa, 
        desde que os participantes comprovem sua matrícula em uma escola pública cadastrada junto ao MEC):
      
        \begin{itemize}
            \item SmallSat: R\$70,00
            \item BigSat: R\$150,00
            \item CanSat: R\$250,00
            \item CubeSat: R\$300,00
        \end{itemize}
        
        \par A necessidade de pagamento da taxa ou isenção para todas as equipes será divulgada
        na mesma data da divuldação dos resultados parciais, 05 de setembro de 2025. Isso não implica
        que os resultados parciais sairão junto com a necessidade de pagamento, mas sim que essa data 
        é determinada para que seja possível haver o maior tempo hábil para busca de patrocinadores e
        doações sem afetar o cronograma da olimpíada.

        \par Reiteramos que esta cobrança ocorrerá apenas em \textbf{último caso}, sendo revogada
        deste edital caso sejam arrecaddados os valores necessários para a realização do evento.
        É previsto aos estudantes que esse seja o único custo em relação ao lancamento. Reforçamos
        também que quaisquer custos adicionais, como alimentação, transporte, estadia, honorários,
        envio de materiais e experimentos pelos correios e etc., ficam a cargo das equipes, o grupo
        Zenith Aerospace se isenta dessas responsabilidades.

        \par As equipes têm total liberdade para realizar campanhas de doação e busca por patrocínios
        para cobrir tais custos extras, mas exigimos que, caso seja feito, seja de forma esteja explícito
        que o destino da arrecadação não será entregue ao grupo Zenith.

        \par Após o encerramento da Olimpíada será divulgado o balanço financeiro proveniennte de 
        doações e patrocínios.

        \par Também é importante ressaltar que, caso o valor arrecadado por meio de doações e patrocínios
        ultrapasse o necessário, esse será destinado a melhorias e investimentos para o grupo Zenith,
        como compra de materiais para pesquisas.