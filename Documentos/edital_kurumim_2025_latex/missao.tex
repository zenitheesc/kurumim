\section{A Missão} 
    A Missão Kurumin é a vertente educacional do grupo Zenith, tendo por objetivo promover e difundir a ciência e engenharia aeroespacial através de experimentos técnicos e científicos lançados nas sondas estratosféricas do grupo. O ambiente estratosférico, devido à suas características ambientais, é perfeito para a emulação de condições espaciais, com radiação, temperatura, pressão e altitude extremas, configurando um excelente ambiente de testes.
    
    Portanto, através do Kurumin, é oferecido aos estudantes dos ensinos Fundamental, Médio, Técnico e Superior, de instituições de ensino públicas e privadas, a oportunidade de desenvolver experimentos nos mais diversos campos do conhecimento (como Biologia, Física, Eletrônica, Telecomunicações, dentre outros) a serem realizados na estratosfera, buscando inspirá-los a seguir o rumo da ciência e engenharia, além envolver-se no setor Aeroespacial.
    
    A missão é totalmente desenvolvida pelo grupo Zenith Aerospace, um grupo de extensão da Escola de Engenharia de São Carlos (USP), que tem por único intuito fomentar a produção científica realizada por seus membros e o desenvolvimento da engenharia aeroespacial no país, sem fins lucrativos. Todos os custos da Olimpíada serão cobertos por doações e patrocínios.
    
    Nesta sétima edição do projeto, teremos como tema \textbf{“Missão Kurumin: Jornada Lunar”}, em homenagem aos 10 anos de fundação do grupo Zenith, criado com o objetivo de desenvolver o satélite Garatéa-L, a ser lançado em órbita lunar. Incentivamos os participantes a desenvolverem um projeto relacionado à temática lunar, em qualquer um de seus aspectos, o importante é a criatividade.
    
    São diversas vertentes de experimentação possíveis: mecânica orbital, vida em situações extremas, comunicação de longa distância, sistemas mecânicos e eletrônicos resistentes a condições espaciais, experimentos bioquímicos no espaço profundo, dentre muitos outros.

    
    \subsection{Modalidades da Olimpíada} \label{sec:modalidades}
        Para facilitar a leitura do edital, a partir deste momento iremos dividir as equipes inscritas em três categorias com base em nível de escolaridade:
        \begin{enumerate}
            \item \textbf{C1:} Ensino Fundamental (I e II);
            \item \textbf{C2:} Ensino Médio ou Técnico;
            \item \textbf{C3:} Ensino Superior.
        \end{enumerate}

        Observe que, para os fins deste edital, o nível de escolaridade corresponde ao que o participante está cursando atualmente. Um aluno no 1º do Ensino Médio corresponderá à categoria C2, por exemplo.
    
        Este ano, a olimpíada é constituída por diferentes modalidades, as quais podem ser selecionadas de acordo com o nível de escolaridade da equipe. Sendo assim, definiremos agora as modalidades em que cada um poderá competir.
        
        \begin{itemize}
            \item \textbf{PseudoSat:} Desenvolvimento de pequenos experimentos individuais abertos à estratosfera, em que o foco se dará exclusivamente no experimento proposto e seu estudo, disponível em dois tamanhos a depender das necessidades da equipe: SmallSat e BigSat. Aberto a todos os níveis de ensino (\textbf{C1, C2 e C3});
            
            \item \textbf{FullSat:} Desenvolvimento de um sistema completo para experimentação na forma de um Cansat ou CubeSat, a depender das necessidades da equipe, incluindo todos os subsistemas necessários para o lançamento em órbita (como comunicação, energia, GPS, sensoriamento, etc). Aberto a todos os níveis, mas pedimos que equipes C1 pensem no alto nível de complexidade envolvido antes de escolher a modalidade.
        \end{itemize}

        Ambas objetivam levar aos alunos uma experiência interdisciplinar e incentivar seu desenvolvimento técnico e científico, além de trazer um enfoque para as diferentes possibilidades de estudo do setor aeroespacial.

    \subsection{Fases}
        A Olimpíada será dividida em fases distintas, cobrindo a concepção do experimento, seu desenvolvimento, sua implementação e a análise final dos dados. Abaixo temos as fases detalhadamente:
        \begin{enumerate}
            \item \textbf{Fase 1 - Planejamento:} Devem ser aqui definidas as equipes e seus nomes, entregando uma visão geral do projeto que será desenvolvido e sua motivação;
            \item \textbf{Fase 2 - Desenvolvimento:} O projeto detalhado na inscrição deve agora ser plenamente desenvolvido, entregando um relatório completo da metodologia empregada;
            \item \textbf{Fase 3 - Lançamento:} Com o experimento construído, este deve ser entregue ao grupo para o lançamento em uma de nossas sondas;
            \item \textbf{Fase 4 - Resultados:} A partir dos resultados obtidos durante o lançamento do experimento, deve ser redigido um relatório final trazendo suas conclusões.
        \end{enumerate}
        Ao longo do edital detalharemos cada fase individualmente.

    \subsection{Doações e Patrocínios}
        Como já destacado, a Missão Kurumin é inteiramente organizada pelo grupo Zenith com intuito de divulgar e promover tecnologias aeroespaciais, sem qualquer fim lucrativo.

        Portanto, para a realização dos lançamentos necessários, altamente custosos, o grupo é dependente de doações e patrocínios para arcar com os custos. Sendo assim, sempre buscamos o auxílio de empresas parceiras e doações individuais para o sucesso da olimpíada. Logo, pedimos ajuda a todos os participantes na divulgação de nossa campanha de arrecadação de custos, disponível no \href{https://www.catarse.me/missao_kurumim_2024}{\color{highcolor}Catarse} (clique para acessar a página).
    
        Caso a campanha e patrocínios não cubram os custos previstos, serão cobradas as seguintes taxas de grupos classificados para a Fase 2 da competição, sendo necessário seu pagamento para a continuidade da equipe na Missão (informamos que grupos de \textbf{escolas da rede pública estarão isentos} desta taxa, desde que os participantes comprovem sua matrícula em uma escola pública cadastrada junto ao MEC):
      
        \begin{itemize}
            \item SmallSat: R\$70,00
            \item BigSat: R\$150,00
            \item CanSat: R\$250,00
            \item CubeSat: R\$300,00
        \end{itemize}
        
        Reiteramos que esta cobrança ocorrerá apenas em último caso, sendo revogada deste edital caso sejam arrecadados custos, por outras fontes, suficientes para o lançamento de todas as equipes. O Zenith prevê que este seja o único custo para a participação dos estudantes, e instituições que cobrem uma taxa além das presentes neste edital para a participação serão automaticamente desclassificadas.