\section{Fase 4 - Resultados}
    Após o lançamento, equipes que estiverem presentes no local receberão seus projetos no mesmo dia para avaliação dos resultados, enquanto as demais terão seus projetos enviados por correio.

    Sendo assim, com o experimento novamente em mãos, inicia-se a fase final da olimpíada, em que os grupos deverão analisar os dados coletados por seu experimento durante o voo e redigir um Relatório Final com os resultados obtidos, a ser enviado ao Zenith até o dia 30/11.

    Sugerimos fortemente que equipes capacitadas busquem reunir os relatórios produzidos em um único de artigo e busquem sua publicação em revistas relevantes. Pedimos apenas que citem a importância do Kurumin no processo.
    
    \subsection{Relatório Final}
        Após o lançamento, as equipes deverão formular um Relatório Final contendo até 10 páginas com os seguintes itens.

        \subsubsection{Capa}
            O Relatório deve contar com uma capa semelhante à Proposta enviada:
            \begin{enumerate}
                \item Título do Trabalho; 
                \item Nome das Instituições de Ensino; 
                \item Nome da Equipe; 
                \item Nome dos Integrantes; 
                \item Nome do Gerente de Projeto;
                \item Nível de Ensino da Equipe;
                \item Local, Mês e Ano. 
            \end{enumerate}
            
        \subsubsection{Resumo}
            Novamente deve ser oferecido um resumo da missão previamente definida, para contextualizar o relatório. Até 200 palavras.
        
        \subsubsection{Metodologia}
            Esta seção objetiva explicar de forma detalhada as técnicas e procedimento empregados na análise dos dados coletados pelo experimento, e como estas permitirão a formulação de uma resposta à missão original.

        \subsubsection{Resultados}
            Devem ser aqui explicitados os resultados alcançados com o experimento, sua confiabilidade e se estes correspondem ao esperado.

        \subsubsection{Conclusão}
            Deve-se analisar o experimento como um todo, avaliando seu sucesso, possíveis melhorias e sua relevância em um contexto científico e/ou social.

        \subsubsection{Referências}
            Toda a bibliografia empregada deve ser citada.

     \subsection{Avaliação}
        A seguir trazemos os critérios de avaliação para o Relatório Final e seus respectivos pesos. A nota aqui obtida, somada às anteriores, determinarão as equipes premiadas pelo Kurumin.
        
        \begin{itemize}
            \item \textbf{Clareza}: Serão avaliadas a organização, clareza e objetividade do relatório (peso 2);
            
            \item \textbf{Metodologia}: Será avaliada a metodologia de análise dos dados coletados, a qual deve possuir fundamentação teórica coerente com os objetivos, estar clara e compreensível (peso 4);

            \item \textbf{Resultados:} Serão avaliados os resultados obtidos com a missão, identificando se estes correspondem à expectativa original e qual é sua relevância social e/ou científica (peso 5);

            \item \textbf{Maestria técnica e científica:} Serão avaliadas os desafios técnicos e científicos enfrentados pela equipe no desenvolvimento da missão e se estes foram superados com sucesso (peso 3).
        \end{itemize}
        
        A cada um desses critérios será atribuída uma nota de 1 a 10, ponderando de acordo com os pesos fornecidos para a média final.

    \subsection{Formulário de Avaliação}
        A cada equipe será enviada também um formulário de avaliação do Kurumin, em que esta poderá prover feedback acerca da organização da competição, impacto do Kurumin na equipe, sugestões para a melhoria do evento, etc.

        Ressalta-se que as informações oferecidas no edital serão de grande ajuda na preparação de edições futuras.